
% JuliaCon proceedings template
\documentclass{juliacon}
\setcounter{page}{1}

%%%
%% Language Support
%%%
\usepackage[english]{babel}
\usepackage{csquotes}

%%%
%% Images
%%%
\usepackage{graphicx}
\graphicspath{{figures/}}
\usepackage[export]{adjustbox}

%%%
%% Fonts and Typewriting
%%%
\usepackage{mathtools}
\usepackage{amssymb}
\usepackage{bbm}
\usepackage{bm}

%%%
%% Enhances Figures and Tables environments
%%%
\usepackage{placeins}
\usepackage[hypcap=true]{caption}
\usepackage{booktabs, multirow, array, tabularx, floatrow, cellspace, threeparttable}
\floatsetup[table]{capposition = top}
\newcolumntype{Z}[1]{>{\centering\arraybackslash}p{#1}}
\newcolumntype{Y}{>{\centering\arraybackslash}X}
\renewcommand{\arraystretch}{1.25}

%%%
%% Enables hyperlinks
%%%
\usepackage{varioref}
\hypersetup{
    pdftitle    = {Econometrics.jl},
    pdfsubject  = {JuliaCon 2019 Proceedings},
    pdfauthor   = {José Bayoán Santiago Calderón},
    pdfkeywords = {Julia, Econometrics, Statistical Software, Software Validation},
    colorlinks  = true,
    linkcolor   = black,
    filecolor   = black,
    urlcolor    = black,
    citecolor   = black
}
\urlstyle{same}

%%%
%% Cross References
%%%
\usepackage{cleveref}

\begin{document}

% **************GENERATED FILE, DO NOT EDIT**************

\title{Econometrics.jl}

\author[]{José Bayoán Santiago Calderón}
\affil[]{Social and Decision Analytics Division, Biocomplexity Institute and Initiative, University of Virginia}

\keywords{Julia, Econometrics, Regression Analysis, Generalized Linear Models, Discrete Choice Models, Instrumental Variables Estimation, Panel Data, Variable Absorption, Software Validation}



\maketitle

\begin{abstract}

Econometrics.jl is a package for econometrics analysis. It provides a series of most common routines for applied econometrics such as models for continuous, nominal, and ordinal outcomes, longitudinal estimators, variable absorption, weights, rank deficient, and robust variance covariance estimators.

\end{abstract}

\section{Introduction}

Software to economists is what hammers are to blacksmiths. As scientists, the tools to work with data are as important as the know-how to perform the analysis. A requirement for good science is that the tools used in the process have a certain standard in terms of quality, transparency, and flexibility. The analysis must allow for the process to be verifiable and replicable.

This study will: (1) provide an overview of common routines and their statistical and technical requirements, (2) provide an overview of tools available with their peculiarities, (3) discuss design decisions in the development of these tools, and (4) showcase \textit{Econometrics.jl} as a new tool for applied work.

Regression models may be used for several purposes. These may provide a basis for prediction models, causal inference, etc. Common targets include confidence intervals of the parameters estimates, joint-significance of a feature, out-of-sample predictive performance, and others. In other words, obtaining the estimates of a model is usually only part of the task. In order to judge a model, one may require to perform diagnostics and tests to justify potential conclusions for an analysis.

Regression analysis is at the core of applied econometrics. It can be challenging to grasp the extent what makes up the broad term of regression analysis. The following describes a very brief survey of what this technique might entail. Regression analysis can be used for both observational and experimental settings and it allows great flexibility for a multitude of applications. The main idea is to find estimates for model parameters to optimize some objective such as the likelihood in maximum likelihood estimation (MLE). Other potential objectives include restricted maximum likelihood (REML) or a Bayesian approach such as maximum a posteriori probability (MAP). One framework is the generalized linear model (GLM) which use a linear predictor that is mapped through a link function to a distribution modeling the response. Continuous responses might use a Normal distribution, count responses a log link with a Negative Binomial distribution, and probability models might use a categorical distribution with links that map to valid probabilities such as the Logit link. In cases such as probability models where the responses are multidimensional, the generalization is known as vector generalized linear models (VGLM). Other generalizations include relaxing the relation between the linear predictor and the outcome to be the sum of smoothing functions through a generalized additive model (GAM) framework or incorporating random effects through a mixed models approach. Some estimators address challenges such as endogeneity, censored responses, and zero-inflated responses through various solutions such as instrumental variables or censored regression model. Others, exploit aspects of the data to overcome challenges or increase efficiency such as random effects in longitudinal data. In relation to the second moment of the estimator, robust variance covariance estimators or boostrapping may be required for inference.

Out of the many potential tools practitioners may require, what are some of the most common? Not every estimator is as widely accessible or commonly used. Some educated guesses may be well justified such as ordinary least squares being more widely used than spatially-weighted regressions. In order to avoid speculation, I defer to a reasonable assumption that the most common estimators are those usually taught in academic programs and available in widely used software similarly to \cite{Renfro_2009}. Most programs teach tools to address the most common response types: continuous, count, rates, nominal, and ordinal outcomes. Hence, some routines might be linear models, Poisson/negative binomial, multinomial logistic regression, and ordinal logistic regression with proportional odds assumption. Topics in time series and panel data are usually offered in most programs. Perhaps the most common topic is short panels (many units of observations through relatively small number of repeated observations). Common estimators include pooling, first-difference, fixed effects / within estimator, and one-way random effects. The between estimator is usually masked as an intermediate model for estimating the error component in the random effects model. Lastly, the two big challenges taught in most programs are endogeneity and heteroscedasticity. These challenges are usually countered through instrumental variables (e.g., 2SLS) or robust variance-covariance estimators (e.g., heteroscedasticity consistent estimators).

\cite{Renfro_2009} surveyed the functionality of 24 alternatives for common econometrics routines. Throughout the history of econometrics software, alternatives have risen and fallen in following. Some high contenders by market share include Stata \cite{Stata}, R \cite{R}, MATLAB, Python \cite{Python}, IBM SPSS Statistics, SAS software, and EViews. These include both commercial and open-source alternatives. Functionality may be provided by the base/standard libraries in the statistical software environment, as a product such as a toolkit or user contributed such as a module/package that is distributed. Some examples of user-contributed functionality include the \textit{reghdfe} Stata module and a series of R packages such as \textit{MASS} \cite{MASS}, \textit{lmtest} \cite{lmtest}, \textit{sandwich} \cite{sandwich}, \textit{plm} \cite{plm}, and \textit{mlogit} \cite{mlogit}.

The Julia language \cite{Julia} is an upcoming language especially well-suited for scientific computing such as econometrics, data science, machine learning, and other related tasks. The following sections describe commons estimators, the Julia ecosystem supporting tools, and Econometrics.jl which provides further functionality.

\section{The JuliaCon Article Class}
\label{sec:documentclass}
%
The juliacon class file preserves the standard LATEX{} interface such
that any document that can be produced using the standard LATEX{}
article class can also be produced with the class file.\vskip 6pt
It is likely that the make up will change after file submission. For
this reason, we ask you to ignore details such as slightly long lines,
page stretching, or figures falling out of synchronization, as these
details can be dealt with at a later stage.\vskip 6pt
Use should be made of symbolic references (\verb \ref ) in order to
protect against late changes of order, etc.

\section{USING THE JuliaCon Article CLASS FILE}

If the file \verb juliacon.cls  is not already in the appropriate system directory
for \LaTeX{} files, either arrange for it to be put there or copy
it to your working directory. The \verb juliacon  document class is implemented
as a complete class, not a document style option. In order to
use the \verb juliacon  document class, replace \verb article  by \verb juliacon  in the
\verb \documentclass  command at the beginning of your document:
\vskip 6pt
\begin{centering}
    \verb \documentclass{article}  \end{centering}
\vskip 6pt
replace by
\vskip 6pt
 \verb \documentclass{juliacon}  \vskip 6pt
In general, the following standard document \verb style  options should
{ \itshape not} be used with the {\footnotesize \itshape article} class file:
\begin{enumerate}
\item[(1)] \verb 10pt,  \verb 11pt,  \verb 12pt   ? unavailable;
\item[(2)] \verb twoside  (no associated style file) ? \verb twoside  is the default;
\item[(3)] \verb fleqn, \verb leqno, \verb titlepage ? should not be used;
\end{enumerate}

\section{Additional Document Style Options}
\label{sec:additional_doc}
%
The following additional style option is available with the \verb juliacon  class file:
\vskip 6pt
Please place any additional command definitions at the very start of
the \LaTeX{} file, before the \verb \begin{document} . For example, user-defined
\verb \def  and \verb \newcommand   commands that define macros for
technical expressions should be placed here. Other author-defined
macros should be kept to a minimum.
\vskip 6pt
Commands that differ from the standard \LaTeX{} interface, or that
are provided in addition to the standard interface, are explained in
this guide. This guide is not a substitute for the \LaTeX{} manual itself.
Authors planning to submit their papers in \LaTeX{} are advised to use
\verb \juliacon.cls  as early as possible in the creation of their files.

%
%
%
%
\begin{table*}[t]
\tabcolsep22pt
\tbl{If necessary, the tables can be extended both columns.}{
\begin{tabular}{|l|l|c|c|}\hline
Label & \multicolumn{1}{c|}{Description}
& Number of Users &
Number of Queries\\\hline
Test 1 & Training Data &
\smash{\raise-7pt\hbox{70}} & 104\\
\cline{1-2}\cline{4-4}
Test 2 & Testing Data I & & 105\\\hline
Test 3 & Testing Data II & 30 & 119\\\hline
& Total & 100 & 328\\\hline
\end{tabular}}
\label{tab:symbols}
\begin{tabnote}
This is an example of table footnote.
\end{tabnote}
\end{table*}
% \begin{figure*}[t]
% \centerline{\includegraphics[width=11cm]{juliagraphs.png}}
% \caption{If necessary, the images can be extended both columns.}
%   \label{fig:sample_image}
% \end{figure*}

\section{Additional features}
\label{sec:additional_faci}
In addition to all the standard \LaTeX{} design elements, the \verb juliacon  class file includes the following features:
In general, once you have used the additional \verb juliacon.cls facilities
in your document, do not process it with a standard \LaTeX{} class
file.

\subsection{Titles, Author's Name, and Affiliation}
\label{subsub:title_auth}
The title of the article, author's name, and affiliation are used at the
beginning of the article (for the main title). These can be produced
using the following code:

\begin{verbatim}
\title{ This is an example of article title} }
\author{
   \large 1st Author \\[-3pt]
   \normalsize 1st author's affiliation  \\[-3pt]
    \normalsize 1st line of address \\[-3pt]
    \normalsize 2nd line of address \\[-3pt]
    \normalsize	1st author's email address \\[-3pt]
  \and
   \large 2nd Author \\[-3pt]
   \normalsize 2nd author's affiliation  \\[-3pt]
    \normalsize 1st line of address \\[-3pt]
    \normalsize 2nd line of address \\[-3pt]
    \normalsize	2nd author's email address \\[-3pt]
\and
   \large 3rd Author \\[-3pt]
   \normalsize 3rd author's affiliation  \\[-3pt]
    \normalsize 1st line of address \\[-3pt]
    \normalsize 2nd line of address \\[-3pt]
    \normalsize	3rd author's email address \\[-3pt]
}
\maketitle
\end{verbatim}

\subsection{Writing Julia code}

A special environment is already defined for Julia code,
built on top of \textit{listings} and \textit{jlcode}.

\begin{verbatim}
\begin{lstlisting}[language = Julia]
using Plots

x = -3.0:0.01:3.0
y = rand(length(x))
plot(x, y)
\end{lstlisting}
\end{verbatim}
\begin{lstlisting}[language = Julia]
using Plots

x = -3.0:0.01:3.0
y = rand(length(x))
plot(x, y)
\end{lstlisting}


\subsection{Abstracts, Key words, term etc...}
\label{subsub:abs_key_etc}

At the beginning of your article, the title should be generated
in the usual way using the \verb \maketitle  command. For genaral tem and keywords use
\verb \terms ,
\verb \keywords  commands respectively. The abstract should be enclosed
within an abstract environment, All these environment
can be produced using the following code:
\begin{verbatim}
\terms{Experimentation, Human Factors}

\keywords{Face animation, image-based modelling...}

\begin{abstract}
In this paper, we propose a new method for the
systematic determination of the model's base of
time varying delay system. This method based on
the construction of the classification data related
to the considered system. The number, the orders,
the time delay and the parameters of the local
models are generated automatically without any
knowledge about the full operating range of the
process. The parametric identification of the local
models is realized by a new recursive algorithm for
on line identification of systems with unknown time
delay. The proposed algorithm allows simultaneous
estimation of time delay and parameters of
discrete-time systems. The effectiveness of
the new method has been illustrated through
simulation.
\end{abstract}

\end{verbatim}

\section{Some guidelines}
\label{sec:some_guide}
The following notes may help you achieve the best effects with the
\verb juliacon  class file.

\subsection{Sections}
\label{subsub:sections}
\LaTeXe{}  provides four levels of section headings and they are all
defined in the \verb juliacon  class file:
\begin{itemize}
\item \verb \section   \item \verb \subsection  \item \verb \subsubsection  \item \verb \paragraph  \end{itemize}
Section headings are automatically converted to allcaps style.
\subsection{Lists}
\label{sec:lists}
%
The \verb juliacon   class file provides unnumbered lists using the
\verb unnumlist  environment for example,

\begin{unnumlist}
\item First unnumbered item which has no label and is indented from the
left margin.
\item Second unnumbered item.
\item Third unnumbered item.
\end{unnumlist}
The unnumbered list which has no label and is indented from the
left margin. was produced by:
\begin{verbatim}
\begin{unnumlist}
\item First unnumbered item...
\item Second unnumbered item...
\item Third unnumbered item...
\end{unnumlist}
\end{verbatim}

The \verb juliacon   class file also provides hyphen list using the
\verb itemize  environment for example,
\begin{itemize}
\item First unnumbered bulleted item which has no label and is indented
from the left margin.
\item Second unnumbered bulleted item.
\item Third unnumbered bulleted item which has no label and is indented
from the left margin.
\end{itemize}
was produced by:
\begin{verbatim}
\begin{itemize}
\item First item...
\item Second item...
\item Third item...
\end{itemize}
\end{verbatim}

Numbered list is also provided in acmtog class file using the
enumerate environment for example,
\begin{enumerate}
\item The attenuated and diluted stellar radiation.
\item Scattered radiation, and
\item Reradiation from other grains.
\end{enumerate}

was produced by:
\begin{verbatim}
\begin{enumerate}
\item The attenuated...
\item Scattered radiation, and...
\item Reradiation from other grains...
\end{enumerate}
\end{verbatim}
\subsection{Illustrations (or figures)}
\label{subsub:sec_Illus}
The \verb juliacon   class file will cope with most of the positioning of
your illustrations and you should not normally use the optional positional
qualifiers on the \verb figure   environment that would override
these decisions.
\vskip 6pt

%
\begin{figure}[t]
\centerline{\includegraphics[width=4cm]{juliagraphs.png}}
\caption{This is example of the image in a column.}
	\label{fig:sample_figure}
\end{figure}

The figure \ref{fig:sample_figure} is taken from the JuliaGraphs
organization \footnote{https://github.com/JuliaGraphs}.

Figure captions should be \emph{below} the figure itself, therefore the
\verb \caption  command should appear after the figure or space left for
an illustration. For example, Figure 1 is produced using the following
commands:

\begin{verbatim}
\begin{figure}
\centerline{\includegraphics[width=20pc]{Graphics.eps}}
\caption{An example of the testing process for a
binary tree. The globa null hypothesis is tested
first at level $\alpha$ (a), and the level of
individual variables is reached last (d). Note
that individual hypotheses can be tested at
level $\alpha/4$ and not $\alpha/8$ as one might
expect at first.}
\label{sample-figure_2}
\end{figure}
\end{verbatim}
Figures can be resized using first and second argument of
\verb \includegraphics   command. First argument is used for modifying
figure height and the second argument is used for modifying
figure width respectively.
\vskip 6pt
Cross-referencing of figures, tables, and numbered, displayed
equations using the \verb \label  and \verb \ref   commands is encouraged.
For example, in referencing Figure 1 above, we used
\verb Figure~\ref{sample-figure}   \subsection{Tables}
\label{subsub:sec_Tab}
The \verb juliacon   class file will cope with most of the positioning of
your tables and you should not normally use the optional positional qualifiers on the table environment which would override these
decisions. Table captions should be at the top.
\begin{verbatim}
\begin{table}
\tbl{Tuning Set and Testing Set}{
\begin{tabular}{|l|l|c|c|}\hline
Label & \multicolumn{1}{c|}{Description}
& Number of Users &
Number of Queries\\\hline
Train70 & Training Data &
\smash{\raise-7pt\hbox{70}} & 104\\
\cline{1-2}\cline{4-4}
Test70 & Testing Data I & & 105\\\hline
Test30 & Testing Data II & 30 & 119\\\hline
& Total & 100 & 328\\\hline
\end{tabular}}
\end{table}\end{verbatim}

\begin{table}
\tbl{Tuning Set and Testing Set}{
\begin{tabular}{|l|l|c|c|}\hline
Label & \multicolumn{1}{c|}{Description}
& Number of Users &
Number of Queries\\\hline
Test 1 & Training Data &
\smash{\raise-7pt\hbox{70}} & 104\\
\cline{1-2}\cline{4-4}
Test 2 & Testing Data I & & 105\\\hline
Test 3 & Testing Data II & 30 & 119\\\hline
& Total & 100 & 328\\\hline
\end{tabular}}
\end{table}
\subsection{Landscaping Pages}
\label{subsub:landscaping_pages}
If a table is too wide to fit the standard measure, it may be turned,
with its caption, to 90 degrees. Landscape tables cannot be produced
directly using the \verb juliacon   class file because \TeX{} itself cannot
turn the page, and not all device drivers provide such a facility.
The following procedure can be used to produce such pages.
\vskip 6pt
Use the package \verb rotating   in your document and change the coding
from
\begin{verbatim}
\begin{table}...\end{table}
to
\begin{sidewaystable}...\end{sidewaystable}
and for figures
\begin{figure}...\end{figure}
to
\begin{sidewaysfigure}...\end{sidewaysfigure}
\end{verbatim}

environments in your document to turn your table on the appropriate
page of your document. For instance, the following code prints
a page with the running head, a message half way down and the
table number towards the bottom.
\begin{verbatim}
\begin{sidewaystable}
\tbl{Landscape table caption to go here.}{...}
\label{landtab}
\end{sidewaystable}
\end{verbatim}

\subsection{Double Column Figure and Tables}
\label{subsub:double_fig_tab}
For generating the output of figures and tables in double column
we can use the following coding:

\begin{enumerate}
\item For Figures:
\begin{verbatim}
\begin{figure*}...\end{figure*}
\end{verbatim}
\item For landscape figures:
\begin{verbatim}
\begin{sidewaysfigure*}...\end{sidewaysfigure*}
\end{verbatim}
\item For Tables:
\begin{verbatim}
\begin{table*}...\end{table*}
\end{verbatim}
\item For landscape tables:
\begin{verbatim}
\begin{sidewaystable*}...\end{sidewaystable*}
\end{verbatim}
\end{enumerate}

\subsection{Typesetting Mathematics}
\label{subsub:type_math}
The \verb juliacon class file will set displayed mathematics with center to
the column width, provided that you use the \LaTeXe{} standard of
open and closed square brackets as delimiters.
The equation
\[
\sum_{i=1}^p \lambda_i = (S)
\]

was typeset using the acmtog class file with the commands

\begin{verbatim}
\[
\sum_{i=1}^p \lambda_i = (S)
\]
\end{verbatim}

For display equations, cross-referencing is encouraged. For example,
\begin{verbatim}
\begin{equation}
(n-1)^{-1} \sum^n_{i=1} (X_i - \overline{X})^2.
\label{eq:samplevar}
\end{equation}
Equation~(\ref{eq:samplevar}) gives the formula for
sample variance.
\end{verbatim}
The following output is generated with the above coding:
\begin{equation}
(n-1)^{-1} \sum^n_{i=1} (X_i - \overline{X})^2.
\label{eq:samplevar}
\end{equation}
Equation~(\ref{eq:samplevar}) gives the formula for
sample variance.


\subsection{Enunciations}
\label{subsub:enunciation}
The \verb juliacon   class file generates the enunciations with the help of
the following commands:
\begin{verbatim}
\begin{theorem}...\end{theorem}
\begin{strategy}...\end{strategy}
\begin{property}...\end{property}
\begin{proposition}...\end{proposition}
\begin{lemma}...\end{lemma}
\begin{example}...\end{example}
\begin{proof}...\end{proof}
\begin{definition}...\end{definition}
\begin{algorithm}...\end{algorithm}
\begin{remark}...\end{remark}
\end{verbatim}
The above-mentioned coding can also include optional arguments
such as
\begin{verbatim}
\begin{theorem}[...]. Example for theorem:
\begin{theorem}[Generalized Poincare Conjecture]
Four score and seven ... created equal.
\end{theorem}
\end{verbatim}

\begin{theorem}[Generalized Poincare Conjecture]
Four score and seven years ago our fathers brought forth,
upon this continent, a new nation, conceived in Liberty,
 and dedicated to the proposition that all men are
created equal.
\end{theorem}


\subsection{Extract}
\label{subsub:extract}
Extract environment should be coded within
\begin{verbatim}
\begin{extract}..\end{extract}
\end{verbatim}

\subsection{Balancing column at last page}
\label{subsub:Balance}
For balancing the both column length at last page use :
\begin{verbatim}
\vadjust{\vfill\pagebreak}
\end{verbatim}

%\vadjust{\vfill\pagebreak}

at appropriate place in your \TeX{} file or in bibliography file.

\section{Handling references}
\label{subsub:references}
References are most easily (and correctly) generated using the
BIBTEX, which is easily invoked via
\begin{verbatim}
\bibliographystyle{juliacon}
\bibliography{ref}
\end{verbatim}
When submitting the document source (.tex) file to external
parties, the ref.bib file should be sent with it.
\cite{bezanson2017julia}

\input{bib.tex}

\end{document}

% Inspired by the International Journal of Computer Applications template
